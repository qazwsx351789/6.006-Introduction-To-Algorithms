%
% 6.006 problem set 1
%
\documentclass[12pt,twoside]{article}

\usepackage{amsmath}
\usepackage{color}

\input{macros}

\setlength{\oddsidemargin}{0pt}
\setlength{\evensidemargin}{0pt}
\setlength{\textwidth}{6.5in}
\setlength{\topmargin}{0in}
\setlength{\textheight}{8.5in}

% Fill these in!
\newcommand{\theproblemsetnum}{1}
\newcommand{\releasedate}{September 8, 2011}
\newcommand{\partaduedate}{Thursday, September 15}
\newcommand{\tabUnit}{3ex}
\newcommand{\tabT}{\hspace*{\tabUnit}}

\begin{document}

\handout{Problem Set \theproblemsetnum}{\releasedate}

\newif\ifsolution
\solutiontrue
\newcommand{\solution}{\textbf{Your Solution:}}

\textbf{Both theory and programming questions} are due {\bf \partaduedate} at {\bf 11:59PM}.
%
Please download the .zip archive for this problem set, and refer to the
\texttt{README.txt} file for instructions on preparing your solutions.
%
Remember, your goal is to communicate. Full credit will be given only
to a correct solution which is described clearly. Convoluted and
obtuse descriptions might receive low marks, even when they are
correct. Also, aim for concise solutions, as it will save you time
spent on write-ups, and also help you conceptualize the key idea of
the problem.

We will provide the solutions to the problem set 10 hours after the problem set
is due, which you will use to find any errors in the proof that you submitted.
You will need to submit a critique of your solutions by \textbf{Tuesday,
September 20th, 11:59PM}. Your grade will be based on both your solutions and
your critique of the solutions.

\setlength{\parindent}{0pt}

\medskip

\hrulefill

\textbf{Collaborators:}
%%% COLLABORATORS START %%%
By myself
%%% COLLABORATORS END %%%

\begin{problems}

\problem \points{15} \textbf{Asymptotic Practice}

For each group of functions, sort the functions in increasing order of
asymptotic (big-O) complexity:

\begin{problemparts}

\problempart \points{5} \textbf{Group 1:}

$$
\begin{array}{rcl}
f_1(n) &=& n^{0.999999} \log n \\
f_2(n) &=& 10000000 n \\
f_3(n) &=& 1.000001^n \\
f_4(n) &=& n^2
\end{array}
$$

\ifsolution \solution{}
%%% PROBLEM 1(a) SOLUTION START %%%
{\color{blue}$f_1(n),f_2(n),f_4(n),f_3(n)$
\\
\\
For $f_1(n)= n^{0.999999} \log n$, since $\log n =O(n^c)$, we can treat constant c as 0.000001,
therefore $f_1(n)= n^{0.999999} \log n = n^{0.999999} \times n^{0.000001} = n = O(n)$.\\
$f_2(n) = 10000000 n=O(n)$.\\
$f_3(n) = 1.000001^n =O(2^n)$\\
$f_4(n) = n^2 = O(n^2)$\\
Consequently, the increasing order of asymptotic complexity is $f_1(n),f_2(n),f_4(n),f_3(n)$.}
\\
%%% PROBLEM 1(a) SOLUTION END %%%
\fi

\problempart \points{5} \textbf{Group 2:}

$$
\begin{array}{rcl}
f_1(n) &=& 2^{2^{1000000}} \\
f_2(n) &=& 2^{100000n} \\
f_3(n) &=& \displaystyle \binom{n}{2} \\
f_4(n) &=& n \sqrt{n}
\end{array}
$$

\ifsolution \solution{}
%%% PROBLEM 1(b) SOLUTION START %%%
{\color{blue}$f_1(n),f_4(n),f_3(n),f_2(n)$
\\
\\
$f_1(n) = 2^{2^{1000000}} = O(1)$\\
$f_2(n) = 2^{100000n} = O(c2^{n}) = O(2^n)$\\
$f_3(n) = \displaystyle \binom{n}{2} = \frac{n!}{2!\times(n-2)!} = \frac{n(n-1)}{2} = O(n^2)$\\
$f_4(n) = n \sqrt{n} = O(n^{\frac{3}{2}})$

Therefore, the increasing order of asymptotic complexity is $f_1(n),f_4(n),f_3(n),f_2(n)$.}
\\
%%% PROBLEM 1(b) SOLUTION END %%%
\fi

\problempart \points{5} \textbf{Group 3:}

$$
\begin{array}{rcl}
f_1(n) &=& n^{\sqrt{n}} \\
f_2(n) &=& 2^n \\
f_3(n) &=& n^{10} \cdot 2^{n / 2} \\
f_4(n) &=& \displaystyle\sum_{i = 1}^{n} (i + 1)
\end{array}
$$

\ifsolution \solution{}
%%% PROBLEM 1(c) SOLUTION START %%%
{\color{blue}$f_4(n),f_1(n),f_3(n),f_2(n)$
\\
\\
$f_1(n) = n^{\sqrt{n}}=(2^{\log_2{n}})^{\sqrt{n}}=O(2^{\sqrt{n}\log_2{n}})$\\
$f_2(n) = 2^n = O(2^n)$\\
$f_3(n) = n^{10} \cdot 2^{n / 2} = O(n^{10}\cdot 2^{n})=2^{10\log_2{n}}\cdot2^{2/n}=O(2^{n/2+\log_2{n}})$\\
$f_4(n) = \displaystyle\sum_{i = 1}^{n} (i + 1) = \frac{(2+(n+1))n}{2}
(\text{summantion of arithmetic sequence}) = O(n^2)$
\\
{\color{red}Due to the equation $(2)^{cg(x)}=(2^c)^g(x)$, where $c$ is a constant and $g(x)$ is a function of $x$,
 which signifies that$f_3(n) = O(\sqrt{2}^{n+2\log_2{n}})$ so $f_3(n)=O(f_2(n))$ but $f_2(n)\neq O(f_3(n))$(special trick)}
Therefore, the increasing order of asymptotic complexity is $f_4(n),f_1(n),f_3(n),f_2(n)$.}
\\
%%% PROBLEM 1(c) SOLUTION END %%%
\fi

\end{problemparts}

\problem \points{15} \textbf{Recurrence Relation Resolution}

For each of the following recurrence relations,
pick the correct asymptotic runtime:

\begin{problemparts}

\problempart \points{5}
Select the correct asymptotic complexity
of an algorithm with runtime $T(n, n)$
where 
$$
\begin{array}{rcll}
T(x, c) &=& \Theta(x) & \textrm{ for $c \le 2$}, \\
T(c, y) &=& \Theta(y) & \textrm{ for $c \le 2$, and} \\
T(x, y) &=& \Theta(x + y) + T(x / 2, y / 2).
\end{array}
$$

\begin{enumerate}
\item $\Theta(\log n)$.
\item $\Theta(n)$.
\item $\Theta(n \log n)$.
\item $\Theta(n \log^2 n)$.
\item $\Theta(n^2)$.
\item $\Theta(2^n)$.
\end{enumerate}

\ifsolution \solution{}
%%% PROBLEM 2(a) SOLUTION START %%%
{\color{blue}2
\\
\\
$T(n,n)=\Theta(n+n)+T(n/2,n/2)=\Theta(2n)+\Theta(n)+\Theta(n/2)+\ldots
+\Theta(1)\\
(\text{sum of geometric sequence})=2n(1-0.5^n)/0.5=4n-n\cdot 2^{(2-n)}=\Theta(n)$}
\\
%%% PROBLEM 2(a) SOLUTION END %%%
\fi

\problempart \points{5}
Select the correct asymptotic complexity
of an algorithm with runtime $T(n, n)$
where 
$$
\begin{array}{rcll}
T(x, c) &=& \Theta(x) & \textrm{ for $c \le 2$}, \\
T(c, y) &=& \Theta(y) & \textrm{ for $c \le 2$, and} \\
T(x, y) &=& \Theta(x) + T(x, y / 2).
\end{array}
$$

\begin{enumerate}
\item $\Theta(\log n)$.
\item $\Theta(n)$.
\item $\Theta(n \log n)$.
\item $\Theta(n \log^2 n)$.
\item $\Theta(n^2)$.
\item $\Theta(2^n)$.
\end{enumerate}

\ifsolution \solution{}
%%% PROBLEM 2(b) SOLUTION START %%%
{\color{blue}3
\\
\\
$T(n,n)=\Theta(n)+T(n,n/2)=\Theta(n)+\Theta(n)+\ldots+\Theta(n)$ ({with $\log (n)$ occurrences of} $\Theta(n)$ in the sum.)$=\log n \cdot\Theta(n)=\Theta(n\log n)$
\\}
%%% PROBLEM 2(b) SOLUTION END %%%
\fi

\problempart \points{5}
Select the correct asymptotic complexity
of an algorithm with runtime $T(n, n)$
where 
$$
\begin{array}{rcll}
T(x, c) &=& \Theta(x) & \textrm{ for $c \le 2$}, \\
T(x, y) &=& \Theta(x) + S(x, y / 2), \\
S(c, y) &=& \Theta(y) & \textrm{ for $c \le 2$, and} \\
S(x, y) &=& \Theta(y) + T(x / 2, y).
\end{array}
$$

\begin{enumerate}
\item $\Theta(\log n)$.
\item $\Theta(n)$.
\item $\Theta(n \log n)$.
\item $\Theta(n \log^2 n)$.
\item $\Theta(n^2)$.
\item $\Theta(2^n)$.
\end{enumerate}

\ifsolution \solution{}
%%% PROBLEM 2(c) SOLUTION START %%%
{\color{blue}2
\\
\\
$T(n,n)= \Theta(n)+S(n,n/2)=\Theta(n)+\Theta(n)+T(n/2,n/2)$, it is similar to the problem (a). From the experience of problem (a). $T(n,n) = \Theta(n)+\Theta(n)+T(n/2,n/2) = \Theta(n)$.
\\}
%%% PROBLEM 2(c) SOLUTION END %%%
\fi

\end{problemparts}

\section*{Peak-Finding}

In Lecture 1,
you saw the peak-finding problem.
As a reminder,
a \emph{peak} in a matrix
is a location with the property that its four neighbors
(north, south, east, and west)
have value less than or equal to the value of the peak.
We have posted Python code for solving this problem
to the website in a file called \texttt{ps1.zip}.
In the file \texttt{algorithms.py},
there are four different algorithms
which have been written
to solve the peak-finding problem,
only some of which are correct.
Your goal is to figure out
which of these algorithms are correct
and which are efficient.

\problem \points{16} \textbf{Peak-Finding Correctness}

\begin{problemparts}

\problempart \points{4} Is \texttt{algorithm1} correct?
\begin{enumerate}
\item Yes.
\item No.
\end{enumerate}

\ifsolution \solution{}
%%% PROBLEM 3(a) SOLUTION START %%%
{\color{blue}Yes.}
%%% PROBLEM 3(a) SOLUTION END %%%
\fi

\problempart \points{4} Is \texttt{algorithm2} correct?
\begin{enumerate}
\item Yes.
\item No.
\end{enumerate}

\ifsolution \solution{}
%%% PROBLEM 3(b) SOLUTION START %%%
{\color{blue}Yes.}
%%% PROBLEM 3(b) SOLUTION END %%%
\fi

\problempart \points{4} Is \texttt{algorithm3} correct?
\begin{enumerate}
\item Yes.
\item No.
\end{enumerate}

\ifsolution \solution{}
%%% PROBLEM 3(c) SOLUTION START %%%
{\color{blue}No.}
%%% PROBLEM 3(c) SOLUTION END %%%
\fi

\problempart \points{4} Is \texttt{algorithm4} correct?
\begin{enumerate}
\item Yes.
\item No.
\end{enumerate}

\ifsolution \solution{}
%%% PROBLEM 3(d) SOLUTION START %%%
{\color{blue}Yes.}
%%% PROBLEM 3(d) SOLUTION END %%%
\fi

\end{problemparts}

\problem \points{16} \textbf{Peak-Finding Efficiency}

\begin{problemparts}

\problempart \points{4} What is the worst-case runtime of \texttt{algorithm1} on a problem of size $n \times n$?
\begin{enumerate}
\item $\Theta(\log n)$.
\item $\Theta(n)$.
\item $\Theta(n \log n)$.
\item $\Theta(n \log^2 n)$.
\item $\Theta(n^2)$.
\item $\Theta(2^n)$.
\end{enumerate}

\ifsolution \solution{}
%%% PROBLEM 4(a) SOLUTION START %%%
{\color{blue}$\Theta(n \log n)$
\\
\\
In algorithm1, the first step is to find the maximum in the middle colume of the problem with $n\times n$ elements , 
which would take $\Theta(n)$. Second, determine whether the neighbors of the maximum is larger than maximum. If so, 
divide problom into subproblem with dimension $n\times2/n$ and repeat the first step until you find a peak.
\\
Consequently, the run time of algorithm1 can be written as $T(n, n) = \Theta(n) + T(n,n/2) = \Theta(n\log n)$, 
which we have discussed in Problem 1-2 (b). Therefore the worst-case runtime of algorithm1 is $\Theta(n\log n)$.}
\\
%%% PROBLEM 4(a) SOLUTION END %%%
\fi

\problempart \points{4} What is the worst-case runtime of \texttt{algorithm2} on a problem of size $n \times n$?
\begin{enumerate}
\item $\Theta(\log n)$.
\item $\Theta(n)$.
\item $\Theta(n \log n)$.
\item $\Theta(n \log^2 n)$.
\item $\Theta(n^2)$.
\item $\Theta(2^n)$.
\end{enumerate}

\ifsolution \solution{}
%%% PROBLEM 4(b) SOLUTION START %%%
{\color{blue}$\Theta(n^2)$
\\
\\
For algorithm2, it utilizes greedy strategy to approach a peak  step by step. Each step costs $\Theta(1)$ runtime.
The worst-case need to check all the elements in the problem. Therefore the worst-case runtime of algorithm1 is 
$\Theta(n^2)$.}
\\
%%% PROBLEM 4(b) SOLUTION END %%%
\fi

\problempart \points{4} What is the worst-case runtime of \texttt{algorithm3} on a problem of size $n \times n$?
\begin{enumerate}
\item $\Theta(\log n)$.
\item $\Theta(n)$.
\item $\Theta(n \log n)$.
\item $\Theta(n \log^2 n)$.
\item $\Theta(n^2)$.
\item $\Theta(2^n)$.
\end{enumerate}

\ifsolution \solution{}
%%% PROBLEM 4(c) SOLUTION START %%%
{\color{blue}$\Theta(n)$
\\
\\
In algorithm3, the first step is to find the maximum in the middle colume and row of the problem with $n\times n$ elements,
which would take $\Theta(2n)$. Second, determine whether the neighbors of the maximum is larger than maximum. If so, 
divide problom into subproblem with dimension $2/n\times2/n$ and repeat the first step until you find a peak.
\\
This algorithm is not correct in this problem. However, the worst-case runtime can be presanted as $T(n,n) = \Theta(2n) +
T(2/n,2/n)$, which we have discussed in Problem 1-2 (a). The worst-case runtime of algorithm3 is $\Theta(n)$.}
\\

%%% PROBLEM 4(c) SOLUTION END %%%
\fi

\problempart \points{4} What is the worst-case runtime of \texttt{algorithm4} on a problem of size $n \times n$?
\begin{enumerate}
\item $\Theta(\log n)$.
\item $\Theta(n)$.
\item $\Theta(n \log n)$.
\item $\Theta(n \log^2 n)$.
\item $\Theta(n^2)$.
\item $\Theta(2^n)$.
\end{enumerate}

\ifsolution \solution{}
%%% PROBLEM 4(d) SOLUTION START %%%
{\color{blue}$\Theta(n)$
\\
\\
In algorithm1, the first step is to find the maximum in the middle colume of the problem with $n\times n$ elements , 
which would take $\Theta(n)$. Second, determine whether the neighbors of the maximum is larger than maximum. If so, 
divide problom into subproblem with dimension $n\times2/n$. Back to first step and, however, find the maximum in the 
middle row of the subproblem then determine whether the neighbors of the maximum is larger than maximum. Do the above 
instructions recursively until we find a peak.
\\
Consequently, the run time of algorithm1 can be written as $T(n, n) = \Theta(n) + S(n,n/2)$, and $S(n, n) = \Theta(n) + T(n/2,n)$,
 which we have discussed in Problem 1-2 (c). Therefore the worst-case runtime of algorithm4 $T(n,n)$ is $\Theta(n)$.}
\\
%%% PROBLEM 4(d) SOLUTION END %%%
\fi

\end{problemparts}

\problem \points{19} \textbf{Peak-Finding Proof}

Please modify the proof below to construct a proof of correctness
for the \emph{most efficient correct algorithm}
among \texttt{algorithm2}, \texttt{algorithm3}, and \texttt{algorithm4}.

The following is the proof of correctness
for \texttt{algorithm1},
which was sketched in Lecture 1.

\begin{quote}
We wish to show that \texttt{algorithm1}
will always return a peak,
as long as the problem is not empty.
To that end,
we wish to prove the following two statements:

{\bf 1. If the peak problem is not empty,
then \texttt{algorithm1} will always return a location.}
Say that we start with a problem of size $m \times n$.
The recursive subproblem examined by \texttt{algorithm1}
will have dimensions
$m \times \lfloor n / 2 \rfloor$ or 
$m \times \left(n - \lfloor n / 2 \rfloor - 1 \right)$.
Therefore, the number of columns in the problem
strictly decreases with each recursive call
as long as $n > 0$.
So \texttt{algorithm1} either returns a location at some point,
or eventually examines a subproblem with a non-positive
number of columns.
The only way for the number of columns to become strictly negative,
according to the formulas that determine the size of the subproblem,
is to have $n = 0$ at some point.
So if \texttt{algorithm1} doesn't return a location,
it must eventually examine an empty subproblem.

We wish to show that there is no way that this can occur.
Assume, to the contrary,
that \texttt{algorithm1} does examine an empty subproblem.
Just prior to this,
it must examine a subproblem of size
$m \times 1$ or $m \times 2$.
If the problem is of size $m \times 1$,
then calculating the maximum of the central column
is equivalent to calculating the maximum of the entire problem.
Hence, the maximum that the algorithm finds must be a peak,
and it will halt and return the location.
If the problem has dimensions $m \times 2$,
then there are two possibilities:
either the maximum of the central column is a peak
(in which case the algorithm will halt and return the location),
or it has a strictly better neighbor in the other column
(in which case the algorithm will recurse
on the non-empty subproblem with dimensions $m \times 1$,
thus reducing to the previous case).
So \texttt{algorithm1} can never recurse into an empty subproblem,
and therefore \texttt{algorithm1} must eventually return a location.

{\bf 2. If \texttt{algorithm1} returns a location,
it will be a peak in the original problem.}
If \texttt{algorithm1} returns a location $(r_1, c_1)$,
then that location must have the best value in column $c_1$,
and must have been a peak within some recursive subproblem.
Assume, for the sake of contradiction,
that $(r_1, c_1)$ is not also a peak within the original problem.
Then as the location $(r_1, c_1)$ is passed up the chain of recursive calls,
it must eventually reach a level where it stops being a peak.
At that level, the location $(r_1, c_1)$
must be adjacent to the dividing column $c_2$ (where $|c_1 - c_2| = 1$),
and the values must satisfy the inequality
$val(r_1, c_1) < val(r_1, c_2)$.

Let $(r_2, c_2)$ be
the location of the maximum value found by \texttt{algorithm1}
in the dividing column.
As a result, it must be that $val(r_1, c_2) \le val(r_2, c_2)$.
Because the algorithm chose to recurse
on the half containing $(r_1, c_1)$,
we know that $val(r_2, c_2) < val(r_2, c_1)$.
Hence, we have the following chain of inequalities:
$$val(r_1, c_1) < val(r_1, c_2) \le val(r_2, c_2) < val(r_2, c_1)$$
But in order for \texttt{algorithm1} to return $(r_1, c_1)$ as a peak,
the value at $(r_1, c_1)$ must have been the greatest in its column,
making $val(r_1, c_1) \ge val(r_2, c_1)$.
Hence, we have a contradiction.
\end{quote}

\ifsolution \solution{}
%%% PROBLEM 5 SOLUTION START %%%
{\color{blue}\\
For algorithm3,  it goes wrong because in the algorithm, it will return 
the result if it finds a maximum in the middle colume and row in the subproblem,
and check the neighbors to ensure they are all smaller than the maximum in the 
subproblem. This instruction may lost the comparison  between the maximum and the 
neighbor outside the subproblem, which result in the incorrect result in some cases.
\\
To fix this instruction, we can compare the the maximum with the best point we have 
ever seen after checking the neighbors around maximum. If the maximum is smaller than 
the best point, do the recursion until the maximum is equal to the best point.
\\}
%%% PROBLEM 5 SOLUTION END %%%
\fi

\problem \points{19} \textbf{Peak-Finding Counterexamples}

For each incorrect algorithm,
upload a Python file giving a counterexample
(i.e. a matrix for which the algorithm returns a location
that is not a peak).

\ifsolution \solution{}
%%% PROBLEM 6 SOLUTION START %%%
{\color{blue}
\begin{verbatim}

Counterexamples for algorithm3

problemMatrix = [
    [0, 0, 0, 0, 0, 0, 0],
    [0, 0, 0, 0, 0, 0, 0],
    [0, 0, 0, 0, 0, 0, 0],
    [0, 3, 0, 0, 0, 1, 0],
    [0, 2, 0, 0, 0, 0, 0],
    [0, 0, 0, 0, 0, 0, 0],
    [0, 0, 6, 5, 0, 0, 0]
]

problemMatrix = [
	[ 4,  5,  6,  7,  8,  7,  6,  5,  4,  3,  2],
	[ 5,  6,  7,  8,  9,  8,  7,  6,  5,  4,  3],
	[ 6,  7,  8,  9, 10, 11,  8,  7,  6,  5,  4],
	[ 7,  8,  9, 10,  9, 10,  9,  8,  7,  6,  5],
	[ 8,  9, 10, 11, 13, 11, 10,  9,  8,  7,  6],
	[ 7,  8,  9, 10, 12, 10,  9,  8,  7,  6,  5],
	[ 6,  7,  8,  9, 10,  9,  8,  7,  6,  5,  4],
	[ 5,  6,  7,  8,  9,  8,  7,  6,  5,  4,  3],
	[ 4,  5,  6,  7,  8,  7,  6,  5,  4,  3,  2],
	[ 3,  4,  5,  6,  7,  6,  5,  4,  3,  2,  1],
	[ 2,  3,  4,  5,  6,  5,  4,  3,  2,  1,  0]
]

\end{verbatim}
}

%%% PROBLEM 6 SOLUTION END %%%
\fi

\end{problems}

\end{document}
